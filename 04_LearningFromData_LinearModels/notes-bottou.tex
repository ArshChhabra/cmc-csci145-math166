\documentclass[10pt]{exam}

\usepackage[margin=1in]{geometry}
\usepackage{amsmath}
\usepackage{amssymb}
\usepackage{amsthm}
\usepackage{mathtools}
\usepackage{bm}
\usepackage{stmaryrd}

\usepackage{booktabs}
\usepackage{array}
\newcolumntype{C}[1]{>{\centering\arraybackslash}p{#1}}

\usepackage{color}
\usepackage{colortbl}
\definecolor{deepblue}{rgb}{0,0,0.5}
\definecolor{deepred}{rgb}{0.6,0,0}
\definecolor{deepgreen}{rgb}{0,0.5,0}
\definecolor{gray}{rgb}{0.7,0.7,0.7}

\usepackage{hyperref}
\hypersetup{
  colorlinks   = true, %Colours links instead of ugly boxes
  urlcolor     = black, %Colour for external hyperlinks
  linkcolor    = blue, %Colour of internal links
  citecolor    = blue  %Colour of citations
}

\usepackage{listings}

%%%%%%%%%%%%%%%%%%%%%%%%%%%%%%%%%%%%%%%%%%%%%%%%%%%%%%%%%%%%%%%%%%%%%%%%%%%%%%%%

\theoremstyle{definition}
\newtheorem{problem}{Problem}
\newtheorem{note}{Note}
\newtheorem{example}{Example}
\newtheorem{defn}{Definition}
\newtheorem{fact}{Fact}
\newtheorem{refr}{References}
\newtheorem{theorem}{Theorem}
\newcommand{\E}{\mathbb E}
\newcommand{\R}{\mathbb R}
\DeclareMathOperator{\nnz}{nnz}
\DeclareMathOperator{\sign}{sign}
\DeclareMathOperator{\determinant}{det}
\DeclareMathOperator{\Var}{Var}
\DeclareMathOperator{\rank}{rank}
\DeclareMathOperator{\prob}{\mathbb P}
\DeclareMathOperator*{\argmin}{arg\,min}
\DeclareMathOperator*{\argmax}{arg\,max}

\newcommand{\Ein}{E_{\text{in}}}
\newcommand{\Eout}{E_{\text{out}}}
\newcommand{\Etest}{E_{\text{test}}}
\newcommand{\I}{\mathbf I}
\newcommand{\Q}{\mathbf Q}
\newcommand{\p}{\mathbf P}
\newcommand{\pb}{\bar {\p}}
\newcommand{\pbb}{\bar {\pb}}
\newcommand{\pr}{\bm \pi}

\newcommand{\trans}[1]{{#1}^{T}}
\newcommand{\loss}{\ell}
\newcommand{\w}{\mathbf w}
\newcommand{\wstar}{{\w}^{*}}
\newcommand{\x}{\mathbf x}
\newcommand{\y}{\mathbf y}
\newcommand{\lone}[1]{{\lVert {#1} \rVert}_1}
\newcommand{\ltwo}[1]{{\lVert {#1} \rVert}_2}
\newcommand{\lp}[1]{{\lVert {#1} \rVert}_p}
\newcommand{\linf}[1]{{\lVert {#1} \rVert}_\infty}
\newcommand{\lF}[1]{{\lVert {#1} \rVert}_F}

\newcommand{\mH}{m_{\mathcal H}}
\newcommand{\dvc}{{d_{\text{VC}}}}
\newcommand{\HH}[1]{\mathcal H_{\text{#1}}}
\newcommand{\Hbinary}{\HH_{\text{binary}}}
\newcommand{\Haxis}{\HH_{\text{axis}}}
\newcommand{\Hperceptron}{\HH_{\text{perceptron}}}


\newcommand{\ignore}[1]{}

%%%%%%%%%%%%%%%%%%%%%%%%%%%%%%%%%%%%%%%%%%%%%%%%%%%%%%%%%%%%%%%%%%%%%%%%%%%%%%%%

\begin{document}


\begin{center}
{
\Huge
Notes: L\'eon Bottou's SGD Paper
}
\end{center}

You are responsible for Sections 1-3 of Bottou's paper \emph{Large-Scale Machine Learning with Stochastic Gradient Descent}.

\section*{Section 3.2, specialized for the logistic loss}

Recall that in logistic regression, we use the logistic loss
\begin{equation}
    Q(z) = \log(1+\exp(-z))
    ,
\end{equation}
where $z = \hat y y = \trans \x \w y$.

\begin{enumerate}
    \item
        What is the gradient of the logistic loss (i.e.\ what is $\nabla_\w Q(z)$)?
        Also, what is the shape of the result and what is the runtime of computing it?
        \vspace{3in}

    \item
        What is the hessian of the logistic loss (i.e.\ what is $\nabla_\w^2 Q(z)$)?
        Also, what is the shape of the result and what is the runtime of computing it?
        \vspace{3in}
\end{enumerate}

\noindent
Table 2 from the paper is reproduced below.
Recall that all stated values are implicitly in big-O notation,
and that the stated run times explicitly ignore dependencies on the number of dimensions $d$ and the cost of computing the $Q$ function.
Use the information from the previous page to make the run times below more precise by including these dependencies.

\vspace{0.15in}
\begin{center}
    \renewcommand*{\arraystretch}{3}
    \begin{tabular}{lC{1in}C{1in}C{1in}C{1in}}
    \toprule
    & GD & 2GD & SGD & 2SGD \\
    \midrule
    Time per iteration: & $n$ & $n$ & 1 & 1 \\
    Iterations to accuracy $\rho$: & $\log\tfrac 1 \rho$ & $\log\log\tfrac 1 \rho$ & $\tfrac1\rho$  & $\tfrac1\rho$  \\
    Time to accuracy $\rho$: & $n\log\tfrac1\rho$ & $n\log\log\tfrac 1 \rho$ & $\tfrac 1\rho$ & $\tfrac 1\rho$ \\
    Time to excess error $\mathcal E$: & $\tfrac 1 {\epsilon^{1/\alpha}} \log^2 \tfrac 1\epsilon$ & $\tfrac1{\epsilon^{1/\alpha}}\log\tfrac 1 \epsilon \log\log \tfrac 1 \epsilon$ & $\tfrac 1 \epsilon$ & $\tfrac 1 \epsilon$ \\
    \bottomrule
\end{tabular}
\end{center}

\ignore{
\section*{Section 2.1, 2.2}
%\begin{problem}
    Reproduce the update formulas for \emph{gradient descent} (GD) and \emph{stochastic gradient descent} (SGD) below.
    This is Equations (2-5) in the reference.
%\end{problem}

\section*{Section 3.1}

Equation (6) states that the excess error $\mathcal E = E(\tilde f_n) - E(f^*)$ can be decomposed in three terms:
\begin{align*}
    \mathcal E 
    & = \bigg(E(f^*_{\mathcal F}) - E(f^*)\bigg)
      + \bigg(E(f_n) - E(f^*_{\mathcal F})\bigg)
      + \bigg(E(\tilde f_n) - E(f_n)\bigg)
\end{align*}

\section*{Section 3.2}
}

\newpage
\section*{Problems}

\begin{problem}
    You are training a logistic regression model using the polynomial feature map with a very high degree,
    and you want your optimization accuracy $\rho$ to be extremely small.
    Which optimization algorithm do you choose and why?
\end{problem}

\vspace{4in}
\begin{problem}
    You are training a logistic regression model.
    Your original dataset had a large number of features and few data points,
    so you applied the PCA feature map to reduce the dimensions in order to ensure good generalization error.
    Which optimization do you choose and why?
\end{problem}

\newpage
\begin{problem}
    You are working at a large social media company,
    and your task is to use logistic regression to predict when a user will click on an ad.
    Your dataset is very large.
    The company has billions of users, and each user has thousands of interactions with ads, and so the number of data points you have is $N>10^{12}$.
    But the number of feature dimension is relatively small, with $d=100$.

    \begin{enumerate}
        \item
            Your boss suggests reducing the size of the dataset and using gradient descent to solve the problem.
            Use VC theory to explain the resulting effect on the excess error of your problem.

        \vspace{4in}
        \item
            Instead of using gradient descent on a sampled dataset, you could use stochastic gradient descent on the original dataset.
            When would this be a good idea?

        \vspace{4in}
        \item
            Your company offers a profit sharing bonus.
            Whenever an employee discovers an algorithm for increasing ad revenue,
            the employee receives 10\% of the resulting increased revenue over the next quarter.
            Last quarter's ad revenue was 1 billion dollars.
            Therefore increasing performance by only 0.1\% will result in the company making 1 million dollars more and a personal bonus of \$100,000.

            Use VC theory to come up with a strategy to increase revenue.
            How will your choice of optimization algorithm change for your new strategy?
    \end{enumerate}
\end{problem}

\clearpage
~
\end{document}



