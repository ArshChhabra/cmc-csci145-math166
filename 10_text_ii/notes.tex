\documentclass[10pt]{article}

\usepackage[margin=1in]{geometry}
\usepackage{amsmath}
\usepackage{amssymb}
\usepackage{amsthm}
\usepackage{mathtools}
\usepackage{bm}
\usepackage{natbib}
\usepackage[inline]{enumitem}

\usepackage{color}
\usepackage{colortbl}
\definecolor{deepblue}{rgb}{0,0,0.5}
\definecolor{deepred}{rgb}{0.6,0,0}
\definecolor{deepgreen}{rgb}{0,0.5,0}
\definecolor{gray}{rgb}{0.7,0.7,0.7}

\usepackage{hyperref}
\hypersetup{
  colorlinks   = true, %Colours links instead of ugly boxes
  urlcolor     = black, %Colour for external hyperlinks
  linkcolor    = blue, %Colour of internal links
  citecolor    = blue  %Colour of citations
}

%%%%%%%%%%%%%%%%%%%%%%%%%%%%%%%%%%%%%%%%%%%%%%%%%%%%%%%%%%%%%%%%%%%%%%%%%%%%%%%%

\theoremstyle{definition}
\newtheorem{problem}{Problem}
\newtheorem{defn}{Definition}
\newtheorem{theorem}{Theorem}
\newtheorem{fact}{Fact}

\newcommand{\R}{\mathbb R}
\DeclareMathOperator{\vcdim}{VCdim}
\DeclareMathOperator{\E}{\mathbb E}
\DeclareMathOperator{\nnz}{nnz}
\DeclareMathOperator{\determinant}{det}
\DeclareMathOperator{\Var}{Var}
\DeclareMathOperator{\rank}{rank}
\DeclareMathOperator*{\argmin}{arg\,min}
\DeclareMathOperator*{\argmax}{arg\,max}
\DeclareMathOperator*{\softmax}{softmax}

\newcommand{\I}{\mathbf I}
\newcommand{\Q}{\mathbf Q}
\newcommand{\p}{\mathbf P}
\newcommand{\pb}{\bar {\p}}
\newcommand{\pbb}{\bar {\pb}}
\newcommand{\pr}{\bm \pi}
\newcommand{\epsapp}{\epsilon_{\text{app}}}
\newcommand{\epsest}{\epsilon_{\text{est}}}

\newcommand{\trans}[1]{{#1}^{T}}
\newcommand{\loss}{\ell}
\newcommand{\aaa}{\mathbf a}
\newcommand{\vv}{\mathbf v}
\newcommand{\uu}{\mathbf u}
\newcommand{\w}{\mathbf w}
\newcommand{\x}{\mathbf x}
\newcommand{\y}{\mathbf y}
\newcommand{\lone}[1]{{\lVert {#1} \rVert}_1}
\newcommand{\ltwo}[1]{{\lVert {#1} \rVert}_2}
\newcommand{\lp}[1]{{\lVert {#1} \rVert}_p}
\newcommand{\linf}[1]{{\lVert {#1} \rVert}_\infty}
\newcommand{\lF}[1]{{\lVert {#1} \rVert}_F}

\newcommand{\dist}[2]{d_{{#1},{#2}}}

\newcommand{\h}{\mathcal H}
\newcommand{\D}{\mathcal D}
\DeclareMathOperator*{\erm}{ERM}

%%%%%%%%%%%%%%%%%%%%%%%%%%%%%%%%%%%%%%%%%%%%%%%%%%%%%%%%%%%%%%%%%%%%%%%%%%%%%%%%

\begin{document}


\begin{center}
\Huge
Notes: Feature Engineering
\end{center}

%%%%%%%%%%%%%%%%%%%%%%%%%%%%%%%%%%%%%%%%%%%%%%%%%%%%%%%%%%%%%%%%%%%%%%%%%%%%%%%%

\noindent
General references:

\begin{enumerate}
    \item \url{https://www.kaggle.com/notebooks?sortBy=voteCount&group=everyone&pageSize=20&datasourceType=competitions}
    \item \url{https://www.kaggle.com/shivamb/extensive-text-data-feature-engineering}
    \item \url{https://www.kaggle.com/sudalairajkumar/getting-started-with-text-preprocessing}
    \item \url{https://towardsdatascience.com/understanding-feature-engineering-part-3-traditional-methods-for-text-data-f6f7d70acd41}
    \item Textbook, Section 25.2
\end{enumerate}

\noindent
\textbf{Important:}
Keep in mind the differences between bayes error, approximation error, and estimation error throughout this discussion.

%%%%%%%%%%%%%%%%%%%%%%%%%%%%%%%%%%%%%%%%

\section{Text}
\begin{problem}
    What is the difference between a \emph{distributed} encoding and a \emph{1-hot} encoding?
\end{problem}

\newpage
\begin{problem}
    A 1-hot encoding of words is sometimes called a \emph{bag of words}.
    What are it's limitations?
    \begin{enumerate}
        %\item Lossy transformation (not bijective)
            %\vspace{3in}
        \item Large dimensionality
            \vspace{3in}
            \newpage
        \item The context problem (i.e., not bijective)
            \vspace{3in}
        \item The phrase problem
            \vspace{3in}
        \item The synonym problem
            \vspace{3in}
        \item The conjugation problem
            \vspace{3in}
        \item The tokenization problem
            \vspace{3in}
        \item The punctuation problem
            \vspace{3in}
        \item The Unicode problem\textbf{S}
            
            See the video ``Unicode and Python: the absolute minimum you need to know'': \url{https://www.youtube.com/watch?v=oXVmZGN6plY}
    \end{enumerate}
\end{problem}

\newpage
\begin{problem}
    What are $n$-grams?
    What are the tradeoffs of using $n$-grams?
\end{problem}

\newpage
\begin{problem}
    What is lemmatization?
    What are the tradeoffs of using lemmatization?
\end{problem}

\newpage
\begin{problem}
    What is text normalization?
    What are the tradeoffs?
\end{problem}

\newpage
\begin{problem}
    What is stop word elimination?
    What are the trade-offs of stop word elimination?
\end{problem}

\newpage
\begin{problem}
    What is the TF-IDF transform?
    What are the tradeoffs?
\end{problem}

\newpage
\begin{problem}
    What is the hashing trick?
    What are the trade-offs of using the hashing trick?

    \noindent
    References:
    \begin{enumerate}
        \item Hashing trick tutorial: \url{https://booking.ai/dont-be-tricked-by-the-hashing-trick-192a6aae3087}
        \item Zipf's law: \url{https://en.wikipedia.org/wiki/Zipf%27s_law}

        \item Excellent research paper on the Johnson-Lindenstrauss lemma: \url{https://papers.nips.cc/paper/7784-fully-understanding-the-hashing-trick}
    \end{enumerate}
\end{problem}

%\begin{problem}
%\end{problem}

%%%%%%%%%%%%%%%%%%%%%%%%%%%%%%%%%%%%%%%%

\newpage
\section{Time}
\begin{problem}
    The discretization transform.
\end{problem}

\newpage
\begin{problem}
    The sin/cos transform.
\end{problem}

%%%%%%%%%%%%%%%%%%%%%%%%%%%%%%%%%%%%%%%%

\newpage
\section{Graph metadata}
\begin{problem}
    Friendship features.
\end{problem}

\newpage
\begin{problem}
    How can pagerank be used in twitter classification?
\end{problem}

%%%%%%%%%%%%%%%%%%%%%%%%%%%%%%%%%%%%%%%%

\newpage
\section{Generic}
\begin{problem}
    What is the clipping transform?
    What are the tradeoffs?
\end{problem}

\newpage
\begin{problem}
    What is the log transform?
    What are the tradeoffs?
\end{problem}

\newpage
\newpage
\begin{problem}
    What is the whitening transform?
    What are the tradeoffs?
\end{problem}

\end{document}


