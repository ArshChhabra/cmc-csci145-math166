\documentclass[10pt]{article}

\usepackage[margin=1in]{geometry}
\usepackage{amsmath}
\usepackage{amssymb}
\usepackage{amsthm}
\usepackage{mathtools}
\usepackage{bm}
\usepackage{natbib}
\usepackage[inline]{enumitem}

\usepackage{color}
\usepackage{colortbl}
\definecolor{deepblue}{rgb}{0,0,0.5}
\definecolor{deepred}{rgb}{0.6,0,0}
\definecolor{deepgreen}{rgb}{0,0.5,0}
\definecolor{gray}{rgb}{0.7,0.7,0.7}

\usepackage{hyperref}
\hypersetup{
  colorlinks   = true, %Colours links instead of ugly boxes
  urlcolor     = black, %Colour for external hyperlinks
  linkcolor    = blue, %Colour of internal links
  citecolor    = blue  %Colour of citations
}

%%%%%%%%%%%%%%%%%%%%%%%%%%%%%%%%%%%%%%%%%%%%%%%%%%%%%%%%%%%%%%%%%%%%%%%%%%%%%%%%

\theoremstyle{definition}
\newtheorem{problem}{Problem}
\newtheorem{defn}{Definition}
\newtheorem{theorem}{Theorem}
\newtheorem{fact}{Fact}

\newcommand{\R}{\mathbb R}
\DeclareMathOperator{\vcdim}{VCdim}
\DeclareMathOperator{\E}{\mathbb E}
\DeclareMathOperator{\nnz}{nnz}
\DeclareMathOperator{\determinant}{det}
\DeclareMathOperator{\Var}{Var}
\DeclareMathOperator{\rank}{rank}
\DeclareMathOperator*{\argmin}{arg\,min}
\DeclareMathOperator*{\argmax}{arg\,max}
\DeclareMathOperator*{\softmax}{softmax}

\newcommand{\I}{\mathbf I}
\newcommand{\Q}{\mathbf Q}
\newcommand{\p}{\mathbf P}
\newcommand{\pb}{\bar {\p}}
\newcommand{\pbb}{\bar {\pb}}
\newcommand{\pr}{\bm \pi}
\newcommand{\epsapp}{\epsilon_{\text{app}}}
\newcommand{\epsest}{\epsilon_{\text{est}}}

\newcommand{\trans}[1]{{#1}^{T}}
\newcommand{\loss}{\ell}
\newcommand{\aaa}{\mathbf a}
\newcommand{\vv}{\mathbf v}
\newcommand{\uu}{\mathbf u}
\newcommand{\w}{\mathbf w}
\newcommand{\x}{\mathbf x}
\newcommand{\y}{\mathbf y}
\newcommand{\lone}[1]{{\lVert {#1} \rVert}_1}
\newcommand{\ltwo}[1]{{\lVert {#1} \rVert}_2}
\newcommand{\lp}[1]{{\lVert {#1} \rVert}_p}
\newcommand{\linf}[1]{{\lVert {#1} \rVert}_\infty}
\newcommand{\lF}[1]{{\lVert {#1} \rVert}_F}

\newcommand{\dist}[2]{d_{{#1},{#2}}}

\newcommand{\h}{\mathcal H}
\newcommand{\D}{\mathcal D}
\DeclareMathOperator*{\erm}{ERM}

%%%%%%%%%%%%%%%%%%%%%%%%%%%%%%%%%%%%%%%%%%%%%%%%%%%%%%%%%%%%%%%%%%%%%%%%%%%%%%%%

\begin{document}


\begin{center}
\Huge
Notes: Multivariate Classification with SGD
\end{center}

%%%%%%%%%%%%%%%%%%%%%%%%%%%%%%%%%%%%%%%%%%%%%%%%%%%%%%%%%%%%%%%%%%%%%%%%%%%%%%%%

\section{References}

\begin{enumerate}
    \item
        The textbook covers multivariate classification in Chapter 17.
    \item
        \url{https://chrisyeh96.github.io/2018/06/11/logistic-regression.html}
\end{enumerate}

\section{Multivariate classification background}

The softmax function is defined to be
\begin{align}
    \softmax &: \R^d \to \R^d \\
    \softmax(\aaa)_i &= \frac {\exp(\aaa_i)}{\sum_{j=1}^d \exp(\aaa_j)}
\end{align}
the softmax function has the important property that it ``squashes'' its input vector into a probability distribution.

The loss function used in multi-class classification is called the ``softmax cross entropy'' loss,
and is the result of composing the softmax function above with the cross entropy
(which is a way to measure the difference of probability distributions).

Assume we are solving a multi-class classification problem with $k$ classes and $d$ input dimensions.
The softmax cross entropy loss is given by
\begin{equation}
    \ell(W;(\x,y)) = - \log \frac {\exp(-\trans\w_y \x)}{\sum_{j=1}^k \exp(-\trans \w_j \x)}
\end{equation}
where for each class $i\in[k]$,
$\w_i : \R^d$ is the parameter vector associated with class $i$;
the variable $W : \R^{k \times d} = (\w_1; \w_2; ...; \w_k)$ is the full parameter matrix;
$\x : \R^d$ is the feature vector;
and $y \in [k]$ is the class label.

\newpage
\noindent\emph{Page intentionally left blank for handwritten notes.}

\newpage
\section{Loss function properties}

\begin{fact}
    The softmax cross entropy function $\loss$ is convex with respect to $W$.
\end{fact}

\begin{fact}
    Assume that $\ltwo{\x}\le \rho$.
    Then softmax cross entropy function $\loss$ is $\rho$-Lipschitz with respect to $W$.
\end{fact}

\begin{fact}
    For each class $i \in [k]$, assume that $\ltwo{\w_i} \le B$.
    Then $\lF{W} \le \sqrt kB$.
\end{fact}
    %\item 
        %All data points in the dataset satisfy $\ltwo{\x} \le \rho$.
        %This implies that the loss function $\ell(W; (\x,y))$ is $\rho$-Lipschitz with respect to $W$.
%\end{enumerate}

\noindent
Theorem 14.8 of Shalev-Shwartz and Ben-David then states that if SGD is run for $T$ iterations to compute parameter estimate $\bar W$,
then
\begin{equation}
    \E L_S(\bar W) - L_S(W^*) \le \frac {\sqrt kB\rho}{\sqrt T}
\end{equation}
where $W^* = \argmin L_S(W)$.

Theorem 14.12 of Shalev-Shwartz and Ben-David then states that if SGD is run for $T$ iterations to compute parameter estimate $\bar W$,
then
\begin{equation}
    \E L_D(\bar W) - L_D(W^*) \le \frac {\sqrt kB\rho}{\sqrt T}
\end{equation}
where $W^* = \argmin L_D(W)$.

\newpage
\noindent\emph{Page intentionally left blank for handwritten notes.}
\newpage
\section{L2 regularization}

Recall that when we apply L2 regularization,
we are minimizing the objective function
\begin{equation}
    f(W) = \tfrac 1 m \sum_{i=1}^m \loss(W; (\x_i,y_i)) + \tfrac\lambda2\lF{W}^2
    .
\end{equation}
This function is $\lambda$-strongly convex in $W$.
(Why?)

\vspace{3in}
\noindent
Then Theorem 14.11 tells us that after performing $T$ iterations of SGD, we have that
\begin{equation}
    \label{eq:f-bound}
    \E f(\bar W) - f(W^*) \le \frac{4\rho^2}{\lambda T} ( 1 + \log T)
\end{equation}
where $W^* = \argmin f(W)$.
Notice that there is no dependence on $k$ in this inequality!

If we substitute $f(W)=L_D(W)+\lF{W}^2$ into Equation \ref{eq:f-bound} above,
then we get
\begin{equation}
    \label{eq:f-bound}
    \E L_D(\bar W) - L_D(W^*) 
    \le \frac{4\rho^2}{\lambda T} ( 1 + \log T) + \tfrac\lambda2\lF{W^*}^2 - \tfrac\lambda2\lF{\bar W}^2
    \le \frac{4\rho^2}{\lambda T} ( 1 + \log T) + \frac{\lambda k B^2}{2}
    .
\end{equation}

\newpage
\section{What to do when $k$ is big?}

A common way to reduce sample complexity when $k$ is large is to factor the parameter matrix as $W = VU$,
where $V : \R ^ {k \times e}$, $U : \R^{e \times d}$, and $e<\!<\!k$.
Then, each $\w_i = \vv_i U$, and the cross entropy softmax loss is
\begin{equation}
    \ell(VU;(\x,y)) = - \log \frac {\exp(-\trans\vv_y U \x)}{\sum_{j=1}^k \exp(-\trans \vv_j U \x)}
    .
\end{equation}

%If $V$ is known a priori, then only the $U$ matrix must be learned,
%and the sample complexity reduces from $O(k\sqrt{T})$ to $O(e/\sqrt{T})$.
%Since $e <\!\!< k$ by assumption, this is a significant improvement.

\begin{fact}
    The softmax cross entropy function $\loss$ is convex with respect to $U$ and $V$.
\end{fact}

\begin{fact}
    Assume that $\ltwo{\x}\le \rho$.
    Then softmax cross entropy function $\loss$ is $\rho$-Lipschitz with respect to $U$ and $V$.
\end{fact}

\begin{fact}
    \begin{enumerate*}[label=(\roman*)]
        \item
    For each class $i \in [k]$, assume that $\ltwo{\vv_i} \le 1$.
    Then, $\lF{V} \le \sqrt k$.
        \item
    Let $\uu_i$ denote the $i$th column of $U$.
    For each column $i \in [e]$, assume that $\ltwo{\uu_i} \le B$.
    Then $\lF{U} \le \sqrt eB$.
        %\item
    %If a matrix $W$ can be factored into the product of matrices $VU$,
    %and $W$ satisfies Fact 3,
    %then $V$ and $U$ can be chosen to satisfy the facts above.
    \end{enumerate*}
\end{fact}

\noindent
Assume that $V$ is fixed and known in advance.
Theorem 14.12 of Shalev-Shwartz and Ben-David then states that if SGD is run for $T$ iterations to compute parameter estimate $\bar U$,
then
\begin{equation}
    \E L_D(V\bar U) - L_D(VU^*) \le \frac {\sqrt eB\rho}{\sqrt T}
\end{equation}
where $U^* = \argmin_U L_D(VU)$.
(A similar result holds for $L_S$ based on Theorem 14.8.)

Assume that $U$ is fixed and known in advance.
Theorem 14.12 of Shalev-Shwartz and Ben-David then states that if SGD is run for $T$ iterations to compute parameter estimate $\bar U$,
then
\begin{equation}
    \E L_D(\bar VU) - L_D(V^*U) \le \frac {\sqrt k\rho}{\sqrt T}
\end{equation}
where $V^* = \argmin_V L_D(VU)$.
(A similar result holds for $L_S$ based on Theorem 14.8.)

\newpage
\section{Open Research Problem (very informal discussion)}

The limitation of the previous technique is that the class labels must have a linear structure.
In this section, we review how to take advantage of arbitrary metric structure.

Let $\mathcal L$ be a metric space of labels,
and $\dist i j$ be the distance between class labels $i$ and $j$.

Build a ``cover tree'' from the class labels.
Let $p_i$ denote the parent label for class $i$.
Then we can rewrite that parameter vector for each class $i$ as
\begin{equation}
    \w_i = \vv_i + \w_{p_i}
    .
\end{equation}
Then the goal is to learn the $\vv_i$ vectors instead of the $\w_i$ vectors.
The matrix $V=(\vv_1,...,\vv_k)$ can be bounded to have size $\sqrt{\dim(\mathcal L)}B$.

It then follows from Theorem 14.12 of Shalev-Shwartz and Ben-David that if SGD is run for $T$ iterations to compute parameter estimate $\bar W$,
then
\begin{equation}
    \E L_D(\bar W) - L_D(W^*) \le \frac {\sqrt {\dim \mathcal L}B\rho}{\sqrt T}
\end{equation}
where $W^* = \argmin L_D(W)$.

%\subsection{Goal}
%
%Some classes
%
%\begin{align}
    %\Pr(Y = i | X) \approx \Pr(Y = j | X) \qquad\text{if}\qquad i \approx j
%\end{align}
%
%\begin{align}
    %&| \log\Pr(Y = a | X) -  \log\Pr(Y = b | X) |
    %\\&= | \trans\w_a \x + \log \sum_j \exp(-\trans\w_j \x) - \trans\w_b - \log \sum_j \exp(-\trans\w_j \x) |
    %\\&= | \trans\w_a \x - \trans\w_b \x |
    %\\&= | (\trans\w_a - \trans\w_b ) \x |
    %\\&\le | \trans\w_a - \trans\w_b | | \x |
    %\\&\le \dist a b | \x |
%\end{align}

\end{document}


